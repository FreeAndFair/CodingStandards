%! program = pdflatexmk

% Free & Fair Coding Standards
% Copyright (C) 2024 Free & Fair

\documentclass[10pt,letter]{article}

%%% PACKAGES
\usepackage[in]{fullpage}
\usepackage{tabularray}
\usepackage[pdftex,bookmarks,colorlinks,breaklinks]{hyperref}
\usepackage{amsmath}
\usepackage{amssymb}
\usepackage{listings}
\usepackage{verbatim}
\usepackage{lmodern}
\usepackage[T1]{fontenc}
\setlength {\marginparwidth}{2cm} 
\usepackage{todonotes}
\usepackage{enumitem}
\hypersetup{linkcolor=blue,citecolor=blue,filecolor=black,urlcolor=blue} 
\parindent 0pt
\textheight 9in
\topmargin 0in
\setlength{\parskip}{10pt plus 1pt minus 1pt}

\usepackage{tocloft} 
\setlength\cftparskip{2pt} 

\renewcommand{\subsectionautorefname}{section}
\renewcommand{\subsubsectionautorefname}{section}

\pagestyle{empty}

%%% BEGIN DOCUMENT
\begin{document}

\begin{center}
{\Large \textbf{Free \& Fair Coding Standards}}
\end{center}
\vspace{-6pt}

\textit{Adapted by Free \& Fair from the KindSoftware Coding Standards} \hspace{\fill} \textit{Version 1.0} \\
\textit{and other offshoots of the original Infospheres Java Coding Standard} \hspace{\fill} \textit{December 2024}

\vspace{-12pt}

\rule{\textwidth}{1pt}

\tableofcontents

\section{Introduction}

Code standards aren't just about obsession, they are about productivity, professionalism, and presentation.
The predecessors to this code standard have been used by several companies, research groups, and numerous individuals including industrial teams at \href{https://www.sun.com/}{Sun Microsystems}, DALi, Fulcrum Microsystems, \href{https://galois.com}{Galois}, and \href{https://certik.com}{CertiK}; the Compositional Computing/Infospheres Group in the Department of Computer Science at \href{https://www.caltech.edu/}{Caltech}; the Security of Systems (SoS) Group in the \href{https://www.ru.nl/en/institute-for-computing-and-information-sciences}{Nijmegen Institute for Computing and Information Science} at the \href{https://www.ru.nl/}{Radboud University Nijmegen}; the KindSoftware: Software Engineering with Applied Formal Methods Group in the \href{https://www.ucd.ie/cs/}{School of Computer Science and Informatics} at \href{https://www.ucd.ie/}{University College Dublin}; the Applied Formal Methods Group in the \href{https://www.tacoma.uw.edu/set}{Institute of Technology} at the \href{https://tacoma.uw.edu}{University of Washington Tacoma}; and students in applied formal methods and introductory programming courses at all the schools listed above and \href{https://www.hmc.edu}{Harvey Mudd College}.

Many people helped with the evolution of this code standard. These practices were not created in a vacuum. They are tried and true rules that have been used to produce both research and commercial systems by teams of dozens of programmers. The resulting products (student assignments, government research project deliverables, open-source projects, internal company tooling, etc.) are readable, maintainable, robust, testable, and professional.

While not all the rules are applicable to every developer or all systems, a general adherence to the rules and a belief in software engineering practices often results in happier and healthier developers and systems.

In the context of this document, ``structure'' means ``organization, look, and feel'' of a body of code. It includes elements of system and subsystem organization (packages), documentation (feature documentation, structured documentation tags such as Javadoc or docstrings, inline comments, and index files), specification (inline and external specifications, running comments, index files), and development (code organization). Adhering to these standards means that every piece of code has the same structure so that a reader can instantly know where to find a piece of information (attribute, import, method, feature, etc.) without having to search through an entire source file.

\section{Documentation}

Documentation is all about communication. If you have any hope or desire to see your code understood and reused, it needs to be comprehensively documented. We use various tools to evaluate the true comment volume of code, depending on implementation language. For most languages a minimal acceptable documentation ratio is about 35\%, but we really look for ratios around 50\%.

These numbers sound high even for extensively documented classroom code, and \emph{incredibly} high for Open Source Software. So where does all this documentation come from?
In-code documentation isn't all that counts toward the ratio. Overviews, design documents, manpages, etc. all contribute as well.

In our experience, once you document a system so that other programmers can both rapidly understand what you have built and be capable of building it from scratch with your design, you will have ratios close to what we have prescribed.

\paragraph{How much is enough?}

Your guiding metric for ``How much is enough?'' should be the following: Communicate exactly enough information in your comments so that the reader can implement what you are describing, correctly.
Quality is better than quantity. Comments that convey no information about the choices made during design or coding, or are overly verbose, are generally not helpful. For example, if a component that can be accessed externally has a variable called \texttt{accessCount} that tracks the number of times it is accessed, a useful comment at that variable’s declaration might be ``The number of external accesses.'' By contrast, a comment stating ``This variable is called \texttt{accessCount} because it counts the number of external accesses and we wanted to name it in a way that made that meaning clear to the eventual reader of this code.'' serves primarily to artificially increase the comment/code ratio. 

\paragraph{Self-documenting code}

Some argue that code should be ``self-documenting''. This is true in some sense, and we enforce it in some ways through naming conventions and running comments, but it isn't enough. What about the reader who doesn't know the language you are coding in? What about proprietary reusable components? Obviously, something more is necessary.
We find this opinion especially among the functional programming languages community. It astounds and confuses us that such a bright group of programmers can seriously think that no documentation is better than good documentation. We're working to make a difference there as well.

\paragraph{Real coders don't document}

Also, for those of you that think that documentation is for weaklings and that real hackers, Masters, and Rambo-coders don't need comments; believe us, we could find some code that you wouldn't understand even with days of head-scratching (if you're in the area, drop by and we'll prove it). Rambo-coders need not apply.
\subsection{Comment Types}
We use \textit{semantic properties} to document program code in regular, parseable, meaningful ways.
Each property has a legal scope of use, called its \textit{context}. Contexts are defined in a coarse, language-independent fashion using inclusion operators in kind theory. Contexts are comprised of \textit{files}, \textit{modules}, \textit{features}, and \textit{variables}.
\textit{Files} are exactly that: data files in which program code resides. The scope of a file encompasses everything contained in that file.
A \textit{module} is some large-scale program unit. Modules are typically realized by an explicit module- or class-like structure. Examples of modules are classes in object-oriented systems, modules in languages of the Modula and ML families, packages in the Ada lineage, etc. Other words and structures typically bound to modules include units, protocols, interfaces, etc.
\textit{Features} are the entry point for computation. Features are often named, have parameters, and return values. Functions and procedures in structured languages are features, as are methods in object-oriented languages, and functions in functional languages.
Finally, \textit{variables} are program variables, attributes, constants, enumerations, etc. Because few languages enforce any access principles for variables, variables can vary in semantics considerably.
We maintain regular conventions for property-based documentation for each of these scopes. They are documented in the next few sections.

\subsubsection{File Comment}

Every file has a file comment containing at least the following information (example below):

\begin{enumerate}
  \item The title of the project with which the file is associated, using the \texttt{title} property.
  \item A short description of the file's contents, using the \texttt{description} property.
  \item Any copyright information relevant to the project, using the \texttt{copyright} property.
  \item The license(s) under which the file is protected, using the \texttt{license} property.
  \item The primary authors of the file, using the \texttt{author} property.
  \item The version of the file, using the \texttt{version} property. 
\end{enumerate}

\subsubsection{Module Comment}

A module (class, interface, package, module, unit, protocol, etc.) comment describes the purpose of the module, guaranteed invariants, usage instructions, and/or usage examples. The beginning of the comment should be a description of the module in exactly one sentence. Also included in the class comment are any reminders or disclaimers about required or desired improvements. See the below chart for appropriate properties and usages.
Each module comment block must include exactly one \texttt{version} property and at least one \texttt{author} property. The \texttt{author} properties often list everyone who has written more than a few lines of the class. Each \texttt{author} property should list exactly one author. The \texttt{version} property's version is a project specific string (e.g. ``1.0b5c7''), preferably following \href{https://semver.org/}{semantic versioning} conventions. Additionally, each module may optionally contain a single-line comment at the end stating that the module declaration has ended.

\subsubsection{Feature Comment}

Use language and local documentation conventions to describe every feature’s nature, purpose, specification, effects, algorithmic notes, usage instructions, reminders, etc. The beginning of a feature comment should be a summary of the feature in exactly one sentence using the \todo{property name here} TODO property. See the chart below for appropriate properties and usages.

\subsubsection{Variable Comment}

At minimum, variables should be commented with those semantic properties that are not implicitly specified by program structure. Many variable-scoped semantic properties are implicit within the structure of the program code given the semantics of the programming language in use. For example, the \texttt{hides} property, when used, is explicit documentation of an implicit feature of the language. Thus, the most important semantic property to use on variables is the \texttt{values} property, which should also be documented as a class invariant.

\section{Naming Conventions}

You can name temporary and loop variables anything you like. We suggest prefixes of ``temp'' for temporary variables and ``loop'' for loop indices.

All other constructs, especially types and attributes of your objects and parameters of modules, must have \textit{some} uniform naming scheme. Different programming languages have different standard conventions, and different engineering teams have their own standard conventions for various languages; we specify some details in the specific language sections near the end of this document.

In general, all variable names should have semantic value. They should not be bound to type or class since that might change over time but the semantics will (should) not.

\section{Semantic Properties}

\textit{Semantic properties} (or just ``properties'' for short) are our generic way of talking about special purpose documentation used to ``tag'' program structures. Examples of documentation properties include Javadoc tags and Eiffel indexing properties. Thus, if you are a Java programmer, read ``property'' as ``tag''; if you are an Eiffel programmer read it as ``index property'' and ``well-structured comment''.
Semantic properties are discussed in two papers: \href{https://www.arxiv.org/abs/cs.SE/0204035}{Semantic Properties for Lightweight Specification in Knowledgeable Development Environments} and \href{https://www.arxiv.org/abs/cs.SE/0204036}{Semantic Component Composition}.

Property values that are strings should be enclosed in double-quotes (''). Strings that are split across lines should use a backslash ($\backslash$) as a continuation character. See the example files and headers for examples.
In all of the following properties, Expression means (predicate | code segment | natural language description). A predicate is a boolean-valued function. Examples of valid predicates for detailed specification purposes include expressions written in specification languages like Formal \href{https://www.bon-method.com/}{BON}, \href{https://www.jmlspecs.org/}{JML}, \href{https://frama-c.com/html/acsl.html}{ACSL}, \href{https://cryptol.net/}{Cryptol}, \href{https://galois.com/project/gumbo/}{GCL}, and \href{https://www.omg.org/spec/OCL/}{OCL}, or traditional predicates in any normal (predicate) calculus or (boolean, Hoare) logic. A code segment is a legal expression from any programming language, but is most often written in the language of the code it documents. A natural language description is a description written in a spoken language (like English).

For all properties, entities enclosed in () or <> must be so enclosed when using the properties. That is, ``(Expression)'' means literally ``('' + the expression + ``)''. Entities enclosed in [ ] in the table are optional, and should not be enclosed in literal square brackets when present.
The table lists, in the ``Usage'' column, the programmatic constructs to which each property is applicable. Possible values for this column are \textit{files}, \textit{modules}, \textit{features}, \textit{variables}, \textit{all}, and \textit{special}. Special usages are explained in the ``Purpose/Description'' column. We provide a map of these constructs to each programming language for which we go into greater detail later in this document.

\subsection{Property Conventions}

The following table documents all properties and their usages.

\begin{longtblr}{colspec={||Q[m,wd=1.6in]|Q[m,wd=0.55in]|Q[m,wd=3.85in]||}, rowhead=1, cells={font=\fontsize{9pt}{10pt}\selectfont}}
\hline
\textbf{Property} & \textbf{Usage} & \textbf{Purpose/Description} \\ \hline
\SetCell[c=3]{c} \textbf{\textit{Meta-Information}} \\* \hline
% -------
\texttt{author Full\_name} & Modules & Lists an author of the module. The text has no special internal structure. A doc comment may contain multiple \texttt{author} properties. A standard \texttt{author} property includes only the full name of the author and should not include other information (like an email address, web page, etc.) unless absolutely necessary. \\ \hline
% -------
{\texttt{refines URL [fully resolved name]} or \\ \texttt{refines Document\_name [Version]}} & Modules, Features & Points to the refinement(s) of this construct, typically via a Git URL plus the fully resolved name of the associated construct.  Alternatively, refers to a relevant document (preferably with a version number) by name. \\ \hline
% -------
\texttt{bug Description} & Modules, Features & Describes a known bug in the module or feature. One property should be used for each bug. If a construct has a known bug, it should be described. Omission of a bug description is considered as bad an offense as the existence of the bug in the first place. \\ \hline
% -------
\texttt{copyright Copyright} & Files & Indicates the holders of copyright in the file. \\ \hline
% -------
\texttt{description Description} & Modules, Features, Files & A brief description of a construct. In the case of a module or feature, this should be a one-sentence summary of the construct; in the case of a file, it can be longer.  This sentence is meant to correspond to a Lando indexing clause for the construct. \\ \hline
% -------
\texttt{history Description} & All & Describes how a feature has been significantly modified over time. For each major change, the description should include: who made it, when it was made (date and version), what was done, why it was done, and a reference to the change request and/or user requirement that resulted in the change being made. \\ \hline
% -------
\texttt{license License} & Files & Indicates the license(s) under which the file is made available. \\ \hline
% -------
\texttt{title Title} & Files & Documents the title of the project with which the file is associated. \\ \hline
% -------
\SetCell[c=3]{c} \textbf{\textit{Pending Work}} \\* \hline
% -------
\texttt{idea Author [Classifier] - Description} & All & Describes an idea for something from the named author. The optional \texttt{Classifier} argument is for ad hoc categorization of ideas. Typical examples include optimizations, algorithmic changes, etc. \\ \hline
% -------
\texttt{review Reviewer - Description} & All & Indicates to a reader that a particular block of code still needs to be reviewed (by the named reviewer) for some reason. Typical reasons include the ``something's not right here'' syndrome, boundary condition checks, not understanding an algorithm completely, etc. \\ \hline
% -------
\texttt{todo Author - Description} & All & Describes some work that still needs to be accomplished by the named author. Typical examples include optimizing algorithms, proving correctness of new code, renaming variables or features to conform to standards, cleaning up code, etc. \\ \hline
% -------
\SetCell[c=3]{c} \textbf{\textit{Formal Specifications}} \\* \hline
% -------
\texttt{[label:] ensures (Expression) [<Exception> -] Description} & Features & Describes a postcondition that holds after a feature has completed successfully. If a postcondition is violated, then \texttt{Exception} (if specified) is raised. One property is used for each postcondition. Note that no requirement is made on the visibility of postconditions, so these are often used as the private, development specifications of a feature. The visibility of a postcondition is exactly the visibility of the feature to which it is bound. Acceptable synonyms for \texttt{ensures} are \texttt{ensure}, \texttt{postcondition}, and \texttt{post}. \\ \hline
% -------
\texttt{generate (Expression) [Description]} & Modules, Features & Describes new, possibly unreferenced, entities (for example, new threads of control) created as a result of the execution of a feature or the instantiation of a module. \\ \hline
% -------
\texttt{invariant (Expression) [<Exception>] Description} & Modules, Features & Specifies a module or feature invariant. An invariant is a predicate (Boolean expression) that must hold whenever the entity being described is in a stable state. \texttt{Expression} must, therefore, evaluate to a Boolean.
An object is defined as being in a stable state when either (a) no threads of control are within the object or operating upon the object, (b) a single thread of control is about to enter a feature of the object, or (c) a single thread of control is about to exit a feature of the object.
If an invariant is violated when an object is in a stable state, \texttt{Exception} (if specified) is raised. \\ \hline
% -------
\texttt{modify (SINGLE-ASSIGNMENT | QUERY | Expression) Description} & All & Indicates the semantics of a construct as follows:
{\\ \texttt{SINGLE-ASSIGNMENT} indicates that the construct will be set or operated upon exactly once. Once the variable is set or the feature is called, it will never be set or called again.
\\ \texttt{QUERY} indicates that the construct is read-only and has no side-effects.
\\ \texttt{Expression} indicates that a feature modifies an object and describes how it does so.} \\ \hline
% -------
\texttt{[label:] requires (Expression) [<Exception> -] [Description]} & Features & Describes a precondition that must hold before the feature can be safely invoked. One property is used for each precondition. If a precondition is violated then \texttt{Exception} (if specified) is raised. The visibility of preconditions follows the same rules as the visibility of postconditions. Acceptable synonyms for \texttt{requires} are \texttt{require}, \texttt{precondition}, and \texttt{pre}. \\ \hline
% -------
\SetCell[c=3]{c} \textbf{\textit{Concurrency Control}} \\* \hline
% -------
\texttt{concurrency (SEQUENTIAL | GUARDED | CONCURRENT | TIMEOUT <value> <Exception> | FAILURE <Exception> | SPECIAL) [Description]} & Modules, Features & Describes the concurrency strategy and/or approach taken by (necessary for) the module or feature. The execution context for a feature should be described at this point. The meanings of the possible parameters are as follows:
{\\ \texttt{SEQUENTIAL} means that callers must coordinate so that only one call or access to the object in question may be outstanding at once. If simultaneous calls occur, the semantics and integrity of the system are not guaranteed.
\\ \texttt{GUARDED} means that multiple calls or accesses from concurrent threads may occur simultaneously to one object in question, but only one is allowed to commence; the others are blocked until the performance of the first operation is complete. This is the behavior of \texttt{synchronized} instance methods in Java.
\\ \texttt{CONCURRENT} indicates that multiple calls or accesses from concurrent threads may occur simultaneously to the object in question, and that all will proceed concurrently with correct semantics. This is the default behavior of Java methods and fields and Eiffel features.
\\ \texttt{TIMEOUT} indicates that if a call to this feature is blocked for a time period greater than or equal to \texttt{<value>}, the exception \texttt{<Exception>} will be raised. The value is specified with units.
\\ \texttt{FAILURE} means that if a call to the feature is currently underway, all additional calls will fail and the exception \texttt{<Exception>} will be raised.
\\ \texttt{SPECIAL} indicates that the feature has concurrency semantics that are not covered by the preceding cases. Make sure to explain the particulars of the feature's semantics in sufficient detail that the reader will be quite clear on your feature's unusual semantics.
\\ }
A feature lacking a concurrency property is considered \texttt{CONCURRENT}. The semantics description is optional on features that are labeled as \texttt{SEQUENTIAL} or \texttt{GUARDED}. In general, all features should have concurrency properties. \\ \hline
% -------
\SetCell[c=3]{c} \textbf{\textit{Usage Information}} \\* \hline
% -------
\texttt{param parameter-name [WHERE (Expression)] Description} & Features & Describes a feature parameter. The description may be continued on the following line(s). The expression indicates restrictions on argument values. Restrictions that result in exceptions, such as \texttt{IllegalArgumentException} in Java, should be indicated as such. In particular, it is important to indicate whether reference arguments are allowed to be \texttt{null}/\texttt{void}. There must be one \texttt{param} property for each and every parameter to the feature. \\ \hline
% -------
\texttt{return Description} & Features & Describes a feature's return value. The description may be continued on the following line(s). If the feature being documented is a simple getter feature (or similar), a \texttt{return} property is still necessary; however, in such a case, the actual feature description may be omitted since the \texttt{return} property completely describes the feature. \\ \hline
% -------
\texttt{exception ExceptionName [IF (Expression)] Description} & Features & Describes an exception that can be raised by the feature, and the circumstances under which it is raised. The guard indicates restrictions on argument values. In particular, it is important to indicate whether reference arguments are allowed to be \texttt{null}/\texttt{void}. There must be one exception property for each and every exception declared to be raised by the feature. In Java, for instance, a \texttt{RuntimeException} or \texttt{IllegalArgumentException} declared in the feature signature must be documented with this tag. It is generally a good idea to declare in the feature signature any runtime exceptions that are raised as a result of conditions controllable at compile time (such as the feature being called in the wrong system state, or a lack of sufficient range checking on passed argument values). \\ \hline
% -------
\SetCell[c=3]{c} \textbf{\textit{Versioning}} \\* \hline
% -------
\texttt{version VersionString} & Files, Modules & Denotes the version of the construct. The text has no special internal structure, though it is encouraged that version numbers follow semantic versioning guidelines. Any construct may contain at most one \texttt{version} property. The \texttt{version} property normally refers to the version of the software project that contains the construct, rather than the version of the construct itself. \\ \hline
% -------
\texttt{deprecated Reference to replacement API.} & All & Indicates that this API element should no longer be used even though it may continue to work. By convention, the text describes the API (or APIs) that replace the deprecated API. Be as specific as necessary to differentiate between overloaded or polymorphic options.
If the API is obsolete and there is no replacement, the argument to \texttt{deprecated} should be "No replacement". \\ \hline
% -------
\texttt{since version-tag} & All & Indicates the (release) version number in which this construct first appeared. This version number, like that for the version property, is usually a project-wide version number rather than a version number for the particular construct in which the documented construct appears. No \texttt{since} property is required for features that have existed since the first version of the project. \\ \hline
% -------
\SetCell[c=3]{c} \textbf{\textit{Inheritance}} \\* \hline
% -------
\texttt{hides FeatureName [Description]} & Variables & Indicates that a particular variable hides a variable in a enclosing context (e.g., a parent class or enclosing module). \\ \hline
% -------
\texttt{overrides FeatureName [Description]} & Features & Indicates that a particular feature overrides a feature in an enclosing context (e.g., a parent class or enclosing module). If the overriding feature does not either call the feature it overrides, or implement a superset of that feature's semantics, its effects must be documented (using \texttt{precondition}, \texttt{postcondition}, and \texttt{invariant} properties and the feature description, as appropriate). \\ \hline
% -------
\SetCell[c=3]{c} \textbf{\textit{Documentation}} \\* \hline
% -------
\texttt{design Description} & All & Provides information about design decisions made about the code and/or useful things to know about parts of the code. \\ \hline
% -------
\texttt{equivalent (Expression | Code Reference)} & Modules, Features & Documents convenience or specialized features that can be defined in terms of a few operations using other features. \\ \hline
% -------
\texttt{example Description} & All & Provides one or more examples of how to use a construct. \\ \hline
% -------
\texttt{see <APIName> Label} & All & { Refers the reader to other related material. Some source processors use this information to generate hyperlinked documentation. \texttt{<APIName>} can be the name of a language API or an HTML anchor.
This property is often used to reference an external document that describes pertinent business rules or information relevant to the source code being documented.
\\
The character `\#' separates the name of a module from the name of one of its features. One of several overloaded features may be selected by including a parenthesized list of argument types after the feature name. Whitespace in \texttt{<APIName>} is significant. If there is more than one argument, there must be a single blank character between the arguments. A comment may contain multiple \texttt{see} properties. } \\ \hline
% -------
\SetCell[c=3]{c} \textbf{\textit{Dependencies}} \\* \hline
% -------
\texttt{references (Expression) [Description]} & Modules, Features, Variables & Indicates that the construct references other constructs like objects, instances, files, etc. This is primarily used to highlight subtle inter-dependencies between modules, features, etc. \\ \hline
% -------
\texttt{trace 
(GitUrl, Name) [Description]} & All & Denotes a traceability relationship between the annotated artifact and another artifact with which it is related.  The nature of the relation—--refinement or equivalence—--is typically clear from context.  The optional \texttt{Description} element is provided in cases where the traceability relation’s meaning is unclear. \\ \hline
% -------
\texttt{use (Expression) [Description]} & All & Indicates exactly those elements that are utilized by this element. This property is primarily used to highlight hidden dependencies. \\ \hline
% -------
\SetCell[c=3]{c} \textbf{\textit{Miscellaneous}} \\* \hline
% -------
\texttt{guard (Expression) [Description]} & Features & Indicates that actions use guarded waits (e.g., Java’s \texttt{Object.wait()}) until the condition specified in \texttt{Expression} holds. \\ \hline
% -------
\texttt{values (Expression) [Description]} & Variables & Describes the possible values of a variable, including ranges and/or distinct values. \\ \hline
% -------
{\texttt{time-complexity (Expression) [Description]} \\ \\ \texttt{space-complexity (Expression) [Description]}} & Features & Documents the time or space complexity of a feature. \texttt{Expression} should be in big-O notation. For example, $O(n*n)$ and $O(n^2)$ are equivalent, but the second is the more common way to express quadratic complexity. The free terms of \texttt{Expression} should be documented in the description; usually, they are related directly to instance variables of the surrounding or related classes or modules. \\ \hline
% -------
\end{longtblr}

\begin{comment}
(DMZ note: I'm commenting out these sections for now, because they're really not in good shape and the tools section, in particular, is somewhat tangential to the purpose of a coding standard.)

\section{Tools}
(DMZ note: this section needs \textit{far} more revision than some of the rest)
\subsection{(X)Emacs}
The canonical summary: \textit{Emacs is the extensible, customizable, self-documenting real-time display editor.}
\textit{(Rewritten from the XEmacs home page)} Emacs is a powerful, extensible text editor with full GUI and terminal support.
Two primary supported free implementations of Emacs exist:
\begin{itemize}
  \item \href{https://www.fsf.org/}{FSF} Emacs, the Free Software Foundation's mainline Emacs implementation, and
  \item \href{https://www.xemacs.org/}{XEmacs}, a highly graphical version initially based on an early version of GNU Emacs 19 from the Free Software Foundation and since kept up to date with recent versions of that product. XEmacs stems from a collaboration of Lucid, Inc. with \href{https://www.sun.com/}{Sun Microsystems, Inc.} and the \href{https://www.cs.uiuc.edu/}{University of Illinois}, with additional support having been provided by \href{https://www.amdahl.com/}{Amdahl Corporation} and \href{https://www.altrasoft.com/}{Altrasoft}.
\end{itemize}

\textbf{Extending Emacs.} (X)Emacs is extended through the use of Emacs Lisp \textit{packages}. The primary packages we use for the development of Java are as follows. This list is by no means complete and reflects our personal biases. Some of the following summaries come directly from the info web pages for the corresponding package.
\begin{itemize}
  \item \textbf{Semantic/Bovine.} Semantic is a program for Emacs which includes, at its core, a lexer, and two compiler compilers (bovinator and wisent). Additional tools include, support for imenu, speedbar, whichfunc, eldoc, hippie-expand, and several other build in tools.\\~The core utility is the parser infrastructure which allows different types of parsers to be linked into the system. Two build in parsers include the ``bovine'' parser, and the ``wisent'' parser.
  \item \textbf{Speedbar.} Speedbar is a program for Emacs which can be used to summarize information related to the current buffer. Its original inspiration is the `explorer' often used in modern development environments, office packages, and web browsers.
  \item \textbf{EIEIO.} Summary.
  \item \textbf{AucTeX.} Summary.
  \item \textbf{Proof General.} Summary.
  \item \textbf{Emacs Library (elib).} Summary.
  \item \textbf{Tramp.} Summary.
  \item \textbf{\href{https://repose.cx/emacs/wiki}{Wiki.}} Emacs-Wiki enables you to create and use hyperlinks and simple formatting in plain text files, and to optionally publish your pages as HTML.
  \item \textbf{\href{https://www.csd.uu.se/\textasciitilde andersl/emacs.shtml}{Folding.}} This package provides a minor mode, compatible with all major editing modes, for folding (hiding) parts of the edited text or program.\\~Folding mode handles a document as a tree, where each branch is bounded by special markers `\{\{\{' and `\}\}\}'. A branch can be placed inside another branch, creating a complete hierarchical structure.
  \item \textbf{Tracker.} This mode keeps a list of projects and associated billing entity. A single project can be selected and it will be marked in a file as active until a new command is made to stop it. A project is automatically terminated when emacs exits. If somehow this fails, the tracking data file is touched at regular intervals, so the next time the tracker is started if there is an open project a reasonable guess can be made as to when to mark it as closed. Each session can have multiple comments associated with it, in which you can mark accomplishments or whatever. It will generate informal daily summary reports and a report of total hours for a particular billee.
  \item \textbf{Calendar and Diary.} Emacs provides the functions of a desk calendar, with a diary of planned or past events. It also has facilities for managing your appointments, and keeping track of how much time you spend working on certain projects.
  \item \textbf{ECB (Emacs Code Browser).} While (X)Emacs already has good \textit{editing} support for many modes, its \textit{browsing} support is somewhat lacking. That's where ECB comes in: it displays a number of informational windows that allow for easy source code navigation and overview.
  \item \textbf{Timeclock.} This mode is for keeping track of time intervals. You can use it for whatever purpose you like, but the typical scenario is to keep track of how much time you spend working on certain projects.
  \item \textbf{Planner/Remember.} Planner mode is an organizer and day planner for Emacs. It helps you keep track of your completed and pending tasks, daily schedule, and notes in plain text files. Planner mode is based on the aforementioned Emacs-Wiki mode.
  \item \textbf{\href{https://cedet.sourceforge.net/ede.shtml}{EDE (Emacs Development Environment).}} EDE provides the gloss that simplifies the learning curve for all the very useful functionality of building and debugging under emacs. In doing so it attempts to emulate a typical IDE (Integrated Development Environment). What this means is that EDE will manage or create your makefiles and other building environment duties so the developer can concentrate on code, and not support files. In doing so, it will be much easier for new programmers to learn and adopt the GNU ways of doing things.
  \item \textbf{\href{https://www.common-lisp.net/project/slime/}{SLIME.}} SLIME is a major mode for programming in Common Lisp. It essentially replaces \textbf{\href{https://sourceforge.net/projects/ilisp/}{ILISP}} mode which is no longer wholly supported. SLIME works with all major variants of Common Lisp.\\~SLIME uses a socket-based communication/RPC interface between Emacs and Lisp. Slime mode is a minor-mode complementing Lisp mode. It includes many commands for interacting with the Common Lisp process, including a a read-eval-print loop and a Common Lisp debugger written in Emacs Lisp. The debugger pops up an Emacs buffer similar to the Emacs/Elisp debugger.
  \item \textbf{CLHS/HyperSpec (Common Lisp HyperSpec).} CLHS and HyperSpec are minor modes used to look up definitions in the ANSI Common Lisp standard (ANSI X3.226-1994) which is \textbf{\href{https://www.lispworks.com/reference/HyperSpec/}{available on the web}}, courtesy of Kent Pitman and Xanalys.
  \item \textbf{\href{https://www.xref-tech.com}{X-Refactory.}} XRef is a cross referencing tool for C and Java. It lets you lookup definitions, uses of variables and functions, and refactor source code in a type-safe way.
  \item \textbf{\href{https://caml.inria.fr/}{Caml and Tuareg mode.}} These are two major modes for programming in (O)Caml. I prefer the former, but many people prefer the latter. It provides a read-eval-print loop, built-in support for compilation, debugging, and more.
  \item \textbf{\href{https://sourceforge.net/projects/preview-latex}{Preview-LaTeX.}} This Elisp/LaTeX package will render your displayed LaTeX equations right into the editing window where they belong. The purpose of preview-latex is to embed LaTeX environments such as display math or figures into the source buffers. By mouse-clicking, you can open the original text. After editing, another click will just run the region in question through LaTeX and redisplay the new results.
  \item \textbf{W3M.} Summary.
  \item \textbf{\href{https://x-symbol.sourceforge.net/}{X-Symbol.}} This file documents X-Symbol, a package providing semi-WYSIWYG for LaTeX, HTML and other ``token languages''. It uses additional fonts and provide input methods to insert their characters into your document.
  \item \textbf{CC-mode.} This is a GNU Emacs mode for editing files containing C, C++, Objective-C, Java, and CORBA IDL code. CC-mode support template insertion, auto-indenting, key shortcuts for often used expressions and blocks and much, much more. Documentation on cc-mode can be found in the info pages of your Emacs process (type control-h then`i').
  \item \textbf{fill-paragraph.} The fill-* functions reformat sections of text, code, etc. to be more readable. I.e. They can automatically reorganize and/or justify your code, text, HTML, etc. to fit your needs. The primary use of fill-paragraph while coding in Java is cleaning up documentation in comments. Most programmers do not like what it does to their source code by default.
  \item \textbf{\href{https://sourceforge.net/projects/oo-browser/}{The OO Browser.}} OOBR, the object-oriented browser, used to be a free piece of software from Bob Weiner. It is/was a very powerful, flexible, multi-language, source code browser (a la the Smalltalk tradition) for Emacs.
  \item \textbf{\href{https://jdee.sunsite.dk/}{JDEE}}\textbf{: The Java Development Environment for Emacs.}: The JDEE is an Emacs Lisp package that provides a highly configurable Emacs wrapper for command-line Java development tools, such as those provided in JavaSoft's JDK. The JDEE provides menu access to a Java compiler, debugger, and API documenter. The Emacs/JDEE combination adds features typically missing from command-line tools, including:
\end{itemize}
\begin{itemize}
  \item syntax coloring
  \item auto indentation
  \item compile error to source links
  \item source-level debugging
  \item source code browsing
\end{itemize}
\begin{itemize}
  \item The JDEE supports both Emacs (Unix and Windows versions) and XEmacs. It is freely available under the GNU public license.\\~Note that much of the functionality of JDEE is simply JDEE using other packages like cc-mode, font-lock-mode, etc. It does pull all of these packages together very well.
  \item \textbf{Tags.} (from XEmacs info documentation node (XEmacs)Tags:) A ``tags table'' is a description of how a multi-file program is broken up into files. It lists the names of the component files and the names and positions of the functions (or other named sub-units) in each file. Grouping the related files makes it possible to search or replace through all the files with one command. Recording the function names and positions makes possible the `M-.' command which finds the definition of a function by looking up which of the files it is in.\\~Briefly, using tags with Java means that your Emacs knows the location and completion of all interesting features (classes, interfaces, fields, methods, etc.) of your project so that you can (a) jump right to the feature definitions, (b) use \textit{completion} (see the next bullet), (c) do regular expression searches and/or replaces on tags (e.g. global replace of a variable), and more. See the documentation on tags in Emacs info for more information on the use of tags.
  \item \textbf{Completion.} When using tags or OOBR, \textit{completion} can be enabled. When enabled, completion will let Emacs complete partially typed names of interesting features like classes, interfaces, methods, etc. No longer do you have to type those 30 character interface names! Emacs does it for you! Do an Emacs apropos search on ``completion'' (control-h `a') for more information. Completion is normally mapped to meta-tab in most modes. (Completion can work even in text mode - it can complete from your local dictionary!)
  \item \textbf{font-lock-mode.} When Font Lock mode is enabled, text is fontified (i.e. put into new fonts, colors, styles, etc. depending upon what you are typing to make the display of information more clear) as you type it. While font locking can make for a more pretty/clear display, it can seriously slow down typing/development for large files and long-lived Emacs sessions. Possible solutions include refining font locking (see turn-on-fast-lock and related functions), Lazy Font Locking (see lazy-shot-mode), caching font lock information (see fast-lock-cache-directories), and more.
  \item \textbf{Jacob - The Java Commando Base.} Jacob is a Java class browser and project manager for Emacs and includes a powerful class wizard. Jacob is a \textit{Java program} and its functionality overlaps with that of JDEE, OOBR, and XEmacs. We find it most useful to use all of these tools in tandem, adjusting to fit personal taste.
\end{itemize}
Our code standard is partially enforced with this \textbf{elisp code} that provides some customizations to CC-mode and more.
\subsubsection{\href{https://www.jmlspecs.org/}{JML}}
The Java Modeling Language (JML) is a behavioral interface specification language and complementary tool that can be used to specify the behavior of Java modules. It combines the approaches of Eiffel and Larch, with some elements of the refinement calculus. We use JML heavily in the development of all of our Java software. In particular, we use \href{https://www.openjml.org/}{OpenJML} (for compilation and extended static checking) and \href{https://github.com/FreeAndFair/JMLUnitNG}{JMLUnitNG} (for unit test generation).
\section{Code Examples}
(DMZ Note: this section also probably needs either a rewrite, or to be eliminated; the code examples are not here at the moment but we can hyperlink them in, or put them inline)
Template code examples are available for the following classifier types. Note that, because you're just going to change them anyway, we elected not to use ``com.kindsoftware'' package names on the samples.
\subsection{Java Examples}
\begin{itemize}
  \item Abstract Class
  \item Normal Class
  \item Exception Class
  \item Interface
\end{itemize}

Also, other code templates are available. The ``CodeTests'' class is used to test our standards for code structure. The ``Javadoc'' class contains a list of all Javadoc tags defined in this document. You can load this up into your editor and cut and paste from it when writing new documentation.
\begin{itemize}
  \item Code Tests class
  \item Javadoc class
\end{itemize}

The Javadoc documentation generated from all of the above classes is also available.
\subsection{Eiffel Examples}
A single class template is available that highlights many of the recommendations made in this code standard.
\end{comment}

\section{Recommendations for Specific Languages}

\subsection{Lando}

To be written.


\subsection{Clafer}

To be written.


\subsection{SysMLv2}

To be written.


\subsection{Tamarin}

To be written.


\subsection{Java}

\definecolor{keywordcolor}{rgb}{0.5,0,0.33}
\definecolor{identifiercolor}{rgb}{0,0,0.75}
\definecolor{commentcolor}{rgb}{0.3,0.3,0.3} 

\lstdefinestyle{custom-java}{language={java},showstringspaces={false},
  basicstyle={\ttfamily\mdseries},
  keywordstyle={\color{keywordcolor}},
  keywordstyle={[2]\color{black}\bfseries},
  keywordstyle={[3]\color{black}\bfseries},
  identifierstyle={\color{identifiercolor}},
  commentstyle={\color{commentcolor}},
  frame=lines}

\lstset{style=custom-java}

The following are Java coding \textit{recommendations.} It is not necessary to follow them to the letter, but we have found it helpful to always keep them in mind. We first give general recommendations, and then specific code style guidelines that can be enforced by static checkers (for which configurations are supplied in the code standard repository\footnote{not yet, but they can/will be}).
%
\subsubsection{General Recommendations}
\label{sec:java:general}

\begin{itemize}
    \item
    Do not use the \texttt{*} form of \texttt{import}. Be precise about what you are importing. Check that all declared imports are actually used.
    
    \emph{Rationale:} Otherwise, readers of your code will have a hard time understanding its context and dependencies. Some people even prefer not using \texttt{import} at all (thus requiring that every class reference be fully dot-qualified), which avoids all possible ambiguity at the expense of requiring more source code changes if package names change; we, however, aren't that freaky.

    \item 
    When sensible, consider writing a \texttt{main()} method for the principal class in each program file. The \texttt{main()} method should provide a simple unit test or demo. For many classes in distributed/networked systems there isn't a means by which you can build self-contained testing, so don't go crazy trying to make a \texttt{main()} method for everything.

    \emph{Rationale:} Forms a basis for testing. Also provides usage examples.

    \item 
    For stand-alone application programs, the class with the \texttt{main()} method should be in a separate source file from those containing ``normal'' classes.
    
    \emph{Rationale:} Hard-wiring an application program in one of its component class files limits the potential for class reuse.

    \item 
    If you can conceive of someone else implementing a class's functionality differently, define an interface rather than an abstract class. Generally, use abstract classes only when they are ``partially abstract''; i.e., they implement some functionality that must be shared across all subclasses.

    \emph{Rationale:} Interfaces are more flexible than abstract classes. They support multiple inheritance and can be used as ``mix-ins'' in otherwise unrelated classes.

    \item
    Consider carefully whether any class should implement the \texttt{Cloneable} and/or \texttt{Serializable} interfaces.

    \emph{Rationale:} These are ``magic'' interfaces in Java; they automatically add possibly-needed functionality only if so requested, but also often require additional work to ensure correct/expected behavior (e.g., overriding the default \texttt{clone()} method to perform a ``deep'' clone). 

    \item 
    Declare a class as \texttt{final} only if it is a subclass or implementation of a class or interface declaring all of its non-implementation-specific methods (a similar principle applies to the declaration of \texttt{final} methods).

    \emph{Rationale:} Making a class \texttt{final} means that no one will ever get a chance to reimplement its functionality. Defining it instead to be a subclass of a base that is not \texttt{final} means that someone at least gets a chance to subclass the base with an alternate implementation, which will essentially always happen in the long run.

    \item
    Never declare instance fields as \texttt{public}.

    \emph{Rationale:} The standard OO reasons. Making fields \texttt{public} gives up control over internal class structure. Also, if fields are \texttt{public}, methods can never assume that they have valid values; there is no way to enforce class invariants in the presence of \texttt{public} fields.

    \item 
    Never rely on implicit initializers for instance fields (such as the fact that reference variables are initialized to \texttt{null}).

    \emph{Rationale:} Minimizes initialization errors.

    \item
    Minimize \texttt{static}s (except for \texttt{static final} constants).

    \emph{Rationale:} \texttt{static} fields act like globals in non-OO languages. They make methods more context-dependent, hide possible side-effects, sometimes present synchronized access problems. and are a source of fragile, non-extensible constructions. Also, neither \texttt{static} fields nor \texttt{static} methods are overridable in any useful sense in subclasses. There are legitimate uses for \texttt{static} fields and methods (e.g., for certain concurrent implementations, \texttt{static} locks or counters that also take advantage of atomics may make sense), but they are few and far between.

    \item 
    Generally prefer \texttt{long} to \texttt{int}, and \texttt{double} to \texttt{float}. But use \texttt{int} for compatibility with standard Java constructs and classes (the primary example is array indexing and all of the things this implies about, for example, the maximum sizes of arrays).

    \emph{Rationale:} Arithmetic overflow and underflow can be 4 billion times less likely with \texttt{long}s than with \texttt{int}s; similarly, fewer precision problems occur with \texttt{double}s than with \texttt{float}s. On the other hand, because of limitations in Java atomicity guarantees, use of \texttt{long}s and \texttt{double}s must be synchronized in some cases where unsynchronized use of \texttt{int}s and \texttt{float}s would be safe.

    \item
    Use \texttt{final} and/or comment conventions (see the \texttt{modifies} and \texttt{values} properties) to indicate whether instance fields that never have their values changed after construction are intended to be constant (immutable) for the lifetime of the object (as opposed to those that just happen not to get assigned in a class, but could in a child class).

    \emph{Rationale:} Access to immutable instance fields generally does not require any synchronization control, but access to mutable fields generally does.

    \item 
    Generally prefer \texttt{protected} to \texttt{private}.

    \emph{Rationale:} Unless you have a good reason for sealing in a particular strategy for using a field or method, you might as well plan for change via subclassing. On the other hand, this almost always entails more work. Basing other code in a base class around \texttt{protected} fields and methods is harder, since you have to either loosen or check assumptions about their properties. (Note that in Java, \texttt{protected} methods are also accessible from unrelated classes in the same package. There is hardly ever any reason to exploit this, though.)

    \item 
    Avoid unnecessary \texttt{public} instance field access and update methods. Write \texttt{public} \texttt{get}/\texttt{set}-style methods only when they are intrinsic aspects of functionality.

    \emph{Rationale:} Most instance fields in most classes must maintain values that are dependent on those of other instance fields. Allowing them to be read or written in isolation makes it harder to ensure that consistent sets of values are always used.

    \item 
    Minimize direct internal access to instance fields inside methods. Use \texttt{protected} access and update methods instead (or sometimes \texttt{public} ones if they exist anyway).

    \emph{Rationale:} While inconvenient and sometimes overkill, this allows you to vary synchronization and notification policies associated with field access and change in the class and/or its subclasses, which is otherwise a serious impediment to extensiblity in concurrent OO programming.

    \item 
    Avoid giving a field the same name as one inherited from a parent class.

    \emph{Rationale:} This is usually an error. If not, there must be a very good reason, which should always be explained with a \texttt{hides} property.

    \item 
    Declare arrays as \texttt{Type[] arrayName} rather than \texttt{Type arrayName[]}.

    \emph{Rationale:} The second form exists solely for incorrigible C programmers, and makes code less easily readable. The use of "[]" is part of the type, so it should be syntactically adjacent to the rest of the type name.

    \item
    Ensure that non-\texttt{private} \texttt{static} fields have sensible values, and that non-\texttt{private} \texttt{static} methods can be executed sensibly, even if no instances are ever created. Use static intitializers (\texttt{static { \ldots }}) if necessary.

    \emph{Rationale:} You cannot assume that non-private statics will be accessed only after instances are constructed.

    \item 
    Write methods that only do ``one thing''. In particular, separate out methods that change object state from those that just rely upon it. A classic example: in a \texttt{Stack} class, prefer having two methods \texttt{Object top()} and \texttt{void removeTop()} to having the single method \texttt{Object pop()} that does both.

    \emph{Rationale:} This simplifies (sometimes, makes even possible) concurrency control and subclass-based extensions.
    
    \item 
    Define return types as void unless they return results that are not (easily) accessible otherwise (i.e., refrain from ever writing \texttt{return this;}).

    \emph{Rationale:} While convenient, the resulting method cascades (\texttt{a.meth1().meth2().meth3()}) can be sources of synchronization problems and other failed expectations about the states of target objects.

    \item 
    Avoid overloading methods on argument type (overriding on arity, as in having a one-argument version versus a two-argument version of a method with the same name, is fine). If you need to specialize behavior according to the class of an argument, consider instead choosing a general type for the nominal argument type (often \texttt{Object}, \texttt{Serializable} or \texttt{Cloneable}) and using conditionals that check \texttt{instanceof} (or, reflectively, \texttt{isAssignableFrom()}). Alternatives include techniques such as double-dispatching or, often best, reformulating methods (and/or their arguments) to remove dependence on exact argument type.

    \emph{Rationale:} Java method resolution is static, based on the source code types at compile time rather than the actual argument types at runtime. This is compounded in the case of non-\texttt{Object} types with coercion charts. In both cases, most programmers have not committed the matching rules to memory. The results can be counterintuitive, and thus the source of subtle errors. For example, try to predict the output of this code. Then compile and run it:

    \begin{lstlisting}
    class Classifier {
          String identify(Object x) {
            return "object";
        }
    
        String identify(Integer x) {
            return "integer";
        }
    }    

    class Relay {
      String relay(Object obj) { 
        return (new Classifier()).identify(obj);
      }
    }

    public class App {
      public static void main(String [] args) {
        Relay relayer = new Relay();
        Integer i = new Integer(17);
        System.out.println(relayer.relay(i));
      }
    }    
    \end{lstlisting}

    \item 
    Declare concurrency property values for at least all \texttt{public} methods. Describe the assumed invocation context and/or rationale for existence or lack of synchronization. If you don't know the concurrency semantics of your method, make it \texttt{synchronized}.

    \emph{Rationale:} In the absence of planning out a set of concurrency control policies, declaring methods as \texttt{synchronized} at least guarantees safety (though not necessarily liveness) in concurrent contexts (every Java program is concurrent to at least some minimal extent). With full synchronization of all methods, the methods may lock up, but the object can never enter into randomly inconsistent states (and thus engage in stupidly or even dangerously wrong behavior) due to concurrency conflicts. If you are worried about efficiency problems due to synchronization, learn enough about concurrent OO programming to plan out more efficient and/or less deadlock-prone policies (i.e., read ``Concurrent Programming in Java'' by Doug Lea).

    \item 
    Prefer synchronized methods to synchronized blocks.

    \emph{Rationale:} Better encsapsulation; less prone to subclassing snags; can be more efficient.

    \item 
    If you override \texttt{Object.equals()}, also override \texttt{Object.hashCode()}, and vice-versa.

    \emph{Rationale:} Essentially all collection classes (sets, maps, etc.) and other utilities that group or compare objects in ways depending on equality rely on the guarantee that, for two objects \texttt{a} and \texttt{b}, \texttt{a.equals(b)} only if \texttt{a.hashCode() == b.hashCode()} (that is, if two objects are equivalent, they must have equal hash codes). If two objects are equivalent but their hash codes don’t match (as they will not with the default \texttt{hashCode()}, which just returns an object ID), collection classes will not work as you expect them to (or, in some cases, at all). While it is a good idea to override \texttt{hashCode()} \emph{well}, so that classes that use them for actual hashing will work efficiently, it suffices for correct functionality to override \texttt{hashCode()} so that it always returns a constant (meaning that any two objects are potentially equivalent and an \texttt{equals()} check must always be done). 

    \item 
    Override \texttt{readObject()} and \texttt{writeObject()} if a \texttt{Serializable} class relies on any state that could differ across processes, including, in particular, hash codes and transient fields.

    \emph{Rationale:} Otherwise, objects of the class will not transport properly.

    \item 
    If you think that \texttt{clone()} may be called on a class you write, then explicitly define it (and declare the class to implement \texttt{Cloneable}).

    \emph{Rationale:} The default shallow-copy version of \texttt{clone()} might not do what you want (and, in fact, does not in most cases).

    \item 
    Always document the fact that a method invokes \texttt{wait()}.

    \emph{Rationale:} Clients may need to take special actions to avoid nested monitor calls.

    \item
    Initialize all (reasonable) fields in all constructors, or do not initialize any fields in any constructors (rely on explicit or implicit initialization in field declarations) and always explicitly call the superclass constructor.

    \emph{Rationale:} If you don't initialize it, you don't know what value it really holds. Consistency is the key in this rule so that a reviewer/user/tool doesn't have to search around for every initialization.

    \item 
    Whenever reasonable, define a default (no-argument) constructor so objects can be created via \texttt{Class.newInstance()}.

    \emph{Rationale:} This allows classes of types unknown at compile time to be dynamically loaded and instantiated. Reflection alleviates the need for no-argument constructors somewhat, but many classes that dynamically instantiate other classes at runtime still depend on their presence.

    \item 
    Prefer abstract methods in base classes to those with default no-op implementations. Also, if there is a common default implementation, consider instead writing it as a \texttt{protected} method so that subclass authors can write a one-line implementation to call the default.

    \emph{Rationale:} The Java compiler will force subclass authors to implement abstract methods, avoiding problems that occur when they forget to do so but should have.

    \item 
    Always use method \texttt{equals()} instead of operator \texttt{==} when comparing objects. In particular, do not use \texttt{==} to compare \texttt{String}s.

    \emph{Rationale:} If someone defined an \texttt{equals()} method to compare objects, then they want you to use it. Otherwise, the default implementation of \texttt{Object.equals()} is just to use \texttt{==}.

    \item 
    Never use \texttt{suspend()}/\texttt{resume()} pairs to implement synchronization with threads.

    \emph{Rationale:} The \texttt{suspend()} and \texttt{resume()} methods are inherently fragile, and were deprecated 
    decades ago for that reason.

    \item 
    Always embed \texttt{wait()} calls in \texttt{while} loops that re-wait if the condition being waited for does not hold.

    \emph{Rationale:} When a \texttt{wait()} wakes up, there is no guarantee that the condition it is waiting for holds (or ever held).

    \item 
    Use \texttt{notifyAll()} instead of \texttt{notify()}.

    \emph{Rationale:} Classes that use only \texttt{notify()} can normally only support at most one kind of wait condition across all methods in the class and all possible subclasses.

    \item 
    Declare a local variable only at that point in the code where you know what its initial value should be.

    \emph{Rationale:} Minimizes bad assumptions about values of variables. Of course, you should know about the initial values of most local variables at the beginning of the method. Only one-off (temporary, loop, etc.) variables should be declared within the body of a method.

    \item 
    Name temporary and loop variables appropriately.

    \emph{Rationale:} Lets the reader immediately know that the variable in question doesn't have a legitimate value unless it is within the code block of its declaration.

    \item 
    Declare and initialize a new local variable rather than reusing (reassigning) an existing one whose value happens to no longer be used at that program point.

    \emph{Rationale:} Minimizes bad assumptions about values of variables.

    \item
    Assign \texttt{null} to any reference variable that is no longer being used. This includes, especially, elements of arrays.

    \emph{Rationale:} Enables the Java garbage collector to do its job. Also, causes fast-failing \texttt{NullPointerException}s (rather than unpredictable behavior) if the code makes bad assumptions about whether reference variables have valid values.

    \item 
    Avoid assignments (``\texttt{=}'') inside \texttt{if} and \texttt{while} conditions.

    \emph{Rationale:} These are almost always typos. The Java compiler catches cases where ``\texttt{=}'' should have been ``\texttt{==}'', except when the variable is (or is automatically coercible to) a \texttt{boolean}.

    \item
    Use \texttt{boolean} values (or values automatically coercible to \texttt{boolean}) directly in \texttt{if} and \texttt{while} conditions; that is, prefer ``\texttt{boolvar}'' to ``\texttt{boolvar == true}'' and ``\texttt{!boolvar}'' to ``\texttt{boolvar == false}''. This also applies to expressions with \texttt{boolean} values.

    \emph{Rationale:} The extra \texttt{==} is redundant, and makes undetectable typos like the ones described in the previous recommendation more likely.

    \item 
    Document cases where the return value of a called method is ignored.

    \emph{Rationale:} These are usually errors. If it is intentional, make the intent clear. A simple way to do this is: \texttt{int unused = obj.methodReturningInt(args);}

    \item
    Ensure that there is ultimately a catch for all unchecked exceptions that can be dealt with.

    \emph{Rationale:} Java allows you to not bother declaring or catching some common easily-handled exceptions, such as \texttt{java.util.NoSuchElementException}. Declare and catch them anyway. It'll make your code more robust.

    \item 
    Embed casts in conditionals. For example: \texttt{C cx = null; if (x instanceof C) cx = (C) x; else evasiveAction();}

    \emph{Rationale:} This forces you to consider what to do if the object is not an instance of the intended class, rather than just generating a \texttt{ClassCastException}.

    \item
    Document fragile constructions used solely for the sake of optimization.

    \emph{Rationale:} These constructions may not be easily understandable to anyone reading your code in the future; also, since they are optimizations based on the behavior of the Java virtual machine, they may need to be updated as Java virtual machine behavior evolves.

    \item
    Do not require 100\% conformance to rules of thumb such as the ones listed here!

    \emph{Rationale:} Java allows you to program in ways that do not conform to these rules for good reason. Sometimes they are the only reasonable ways to implement things. And some of these rules make programs less efficient than they might otherwise be, so they are meant to be conscientiously broken when performance is an issue.

\end{itemize}

\subsubsection{Style Guidelines}

Unlike the general recommendations of \autoref{sec:java:general}, which we encourage \emph{all} Java programmers working on any type of project to follow except when there is good reason not to, the following style guidelines are optional. We use them on our projects for the reasons described, but different development teams have different styles and different sensibilities, and they may not work for everyone. 

That being said, these recommendations are phrased as normative guidelines; for our internal development work, they are.

\paragraph{Tabs, Spacing and Braces}

\begin{itemize}

    \item 
    The ``tab'' character (\texttt{\textbackslash{}t}) may not appear in any source file, unless it is part of a string literal (that is, you define a string that contains a ``\texttt{\textbackslash{}t}'' in it for user interaction or other purposes). 

    \emph{Rationale:} Tab characters
lead to inconsistent code appearance in different editors/IDEs; in
addition, because tabs are invisible, it is impossible for you to tell
whether what you have at the beginning of a line is (for example) 6
spaces or 3 tab characters.

    \item 
    The standard indentation width is 2 spaces. 
    
    \emph{Rationale:} It is the most
commonly used standard indentation width and strikes a nice balance
between use of horizontal space and logical delineation of code
blocks.

    \item 
    Spacing is enforced between operators and operands (i.e., ``\texttt{a +
  b}'' rather than ``\texttt{a+b}''), between commas and entities that
follow them (i.e., ``\texttt{int the\_x, boolean the\_y}'' rather than
``\texttt{int the\_x,boolean the\_y}''), and in a number of other
places. The exception is parentheses, such as in method parameter
lists (i.e., ``\texttt{public void foo()}'' rather than ``\texttt{public
  void foo ( )}''. 
  
    \emph{Rationale:} This spacing generally makes identifiers
easier to see, and code therefore easier to read.

    \item 
    Brace placement can be done in either of the two commonly-accepted
ways: with the opening brace on the same line as a
declaration/statement (\autoref{fig:sameline}) or with the opening brace on the next
line (\autoref{fig:nextline}). Within a given project, only one of these two styles may be used. 


    \begin{figure}
    \begin{center}
    \begin{minipage}{0.5\textwidth}
    \small
    \begin{lstlisting}
      public void m() {
        if (my_flag) {
          // do something
        } else { 
          // do something else
        }
      }
    \end{lstlisting}
    \vspace{-12pt}
    \end{minipage}
    \end{center}
    \caption{The ``same line'' method of brace placement.}
    \label{fig:sameline}
    \vspace{18pt}
    \end{figure}
    
    \begin{figure}
    \begin{center}
    \begin{minipage}{0.5\textwidth}
    \small
    \begin{lstlisting}
      public void m()
      {
        if (my_flag)
        {
          // do something
        } 
        else
        {
          // do something else
        }
      }
    \end{lstlisting}
    \vspace{-12pt}
    \end{minipage}
    \end{center}
    \caption{The ``next line'' method of brace placement.}
    \label{fig:nextline}
    \end{figure}
    
    \emph{Rationale:} Consistency makes code easier to comprehend.

CheckStyle files and Eclipse format files are\footnote{Actually ``will be,'' if we end up deciding to use these tools.} provided for both brace
styles; they may be modified to change the standard indentation width
from 2 to another value, but no other changes to the spacing
guidelines are permitted.

\end{itemize}

\paragraph{Usage of \texttt{final} Modifier} 
    The Java \texttt{final} modifier must be used on every method
parameter (unconditionally), as well as on every field and local
variable that will in fact remain unchanged during execution. Static analysis tools using our configurations\footnote{Again, if we decide we're using them on our project.} will warn you if the \texttt{final}
modifier has been omitted from a field that appears
final.

Note that because static analysis tools cannot predict how you
will use fields and local variables in code you have not written yet,
you may see a lot of \texttt{final}-related warnings early. You can
safely ignore warnings about non-\texttt{final} fields and local
variables when you have not yet written all the code that uses those
fields and local variables.

    \textit{Rationale:} by marking method parameters \texttt{final}, you completely
eliminate bugs that result from accidentally assigning values to them
(something that is almost never intentional, and is always
unintuitive). Also, every \texttt{final} field and local variable is
one that you can effectively ignore when debugging your code.

\paragraph{Method, Field, Parameter and Variable Naming}

The following rules apply to the naming of various Java entities. In
all cases, field names should be as self-documenting as possible. This
may involve making them longer than you're used to; that's OK, since
modern IDEs have auto-completion and modern compilers and runtime
environments don't care about identifier lengths.
%
\begin{itemize}
\item Methods are named \texttt{inCamelCase}; that is, according to
  standard Java convention.
\item \texttt{static final} fields (constants) are named
  \texttt{IN\_ALL\_CAPS\_WITH\_UNDERSCORES}; that is, according to
  standard Java convention. However, their names may not start with
  \texttt{MY\_}, \texttt{THE\_}, or any of the other ``special''
  prefixes described below.
\item Instance fields are named like \texttt{my\_fie1d\_nam3}; that
  is, all lowercase letters and numbers, starting with \texttt{my\_}
  and with ``words'' separated by underscores.\footnote{Note that
    ``my\_fie1d\_nam3'', and similar ``l33tsp3ak'' field names
    (\url{http://en.wikipedia.org/wiki/L33t}), are generally not good
    choices; this example is meant solely to illustrate that numbers
    are allowed in instance field names.}
\item Method parameters are named like
  \texttt{a/an/the/some\_name} or \texttt{thing\_1/2/3}; that is,
  all lowercase letters and numbers with ``words'' separated by
  underscores, either starting with \texttt{a\_}, \texttt{an\_},
  \texttt{the\_} or \texttt{some\_} or ending with a single-digit
  number (1--9).\footnote{You can't go higher than 9, but you
    shouldn't have that many method parameters anyway.}
\item Local variables are named \texttt{variable\_name}; that is, just
  like instance fields but with no leading \texttt{my\_} (they also
  may not start with \texttt{a\_}, \texttt{an\_}, \texttt{the\_} or
  \texttt{some\_}, to differentiate them from method parameters).
\item Except for \texttt{static final} field names and method names,
  uppercase letters are \emph{never} used in Java identifiers.
\end{itemize}
%
\emph{Rationale:} There are several reasons for this naming scheme. First, it
prevents you from accidentally shadowing variables since local
variables, method parameters, and instance fields are forbidden from
having identical names. Second, it makes it obvious when you are
manipulating object state (\texttt{my\_}, indicating that a field
``belongs to'' its containing object) or static state (\texttt{CAPS}),
which can affect later method calls or other threads, rather than
local state, which cannot affect anything other than the current
thread. Finally, it makes object-oriented analysis easier (the field
and parameter names flow well in English) and makes debugging easier
(you can immediately identify what kind of state an identifier
represents).

\paragraph{Documentation}

\emph{Every} class, method, and field---not just the \texttt{public}
ones---must have \emph{complete} Javadoc documentation. For more
information on Javadoc, see the Oracle tutorial on writing Javadoc
comments at
\url{http://java.sun.com/j2se/javadoc/writingdoccomments/index.html}. 

In addition to full Javadoc, every source file should have a
non-Javadoc header comment identifying it as belonging to a specific
project/fileset. For example, the code
shown in \autoref{fig:header} would be a reasonable start for a class
called \texttt{Example} (this also obeys the import rules described below).

\begin{figure}[t]
\begin{center}
\begin{minipage}{0.95\textwidth}
\small
\begin{lstlisting}
/*
 * Java Naming and Style Guidelines
 * Copyright (C) 2011-13 Daniel M. Zimmerman
 * Copyright (C) 2024 Free & Fair
 * See "LICENSE.md" for license information
 */

package mypackage;

import java.util.List;
import java.util.Set;

import javax.swing.JFrame;

import mypackage.util.UsefulClass;

/**
 * Example is a class that conforms to the Naming and 
 * Style Guidelines for Java you are reading right now.
 *
 * @author Daniel M. Zimmerman
 * @version 1.5
 */
public class Example
{
  // code for class Example goes here
}
\end{lstlisting}
\end{minipage}
\vspace{-12pt}
\end{center}
\caption{A conforming source file, including imports and class
  Javadoc but no code.}
\label{fig:header}
\vspace{12pt}
\end{figure}

\emph{Rationale:} Everything has to be documented somehow. Having a
consistent header comment across all files in a project makes it easy
to identify files as part of that project, and provides a single point
of reference for copyright/license information. Using Javadoc for all
in-code documentation enforces consistency within that documentation
and reduces the amount of time you have to spend deciding whether
entities need Javadoc comments. It also means that you don't have to
rewrite/modify comments when you change protection levels of entities
in your code during refactoring.

\paragraph{Import and Package Organization}

Import and package statements should be organized for readability,
according to the following rules. Most IDEs will do this organization
for you automatically.
%
\begin{itemize}
\item The package statement, if any, appears immediately after the
  header comment of each source file and before any import statements.
\item All import statements appear after the package statement (or
  header comment, in the absence of a package statement) of each
  source file and before the class Javadoc comment.
\item Groups of imports are separated by a single blank line.
\item Imports are listed in alphabetical order within their groups.
\item Imports from \texttt{java.} packages, if any, are declared in the first group.
\item Imports from \texttt{javax.} packages, if any, are declared in the next group.
\item Other imports are declared in the next group (you can optionally
  make multiple groups of these, as you see fit, but this should be done consistently across all source files in the project).
\end{itemize}

\emph{Rationale:} Consistency is important for understanding code, and this applies to import declarations as well. Given the general rule about not using the ``*'' form of \texttt{import}, this means that every source file in a project will list all its dependencies in the same, well-defined order.

\paragraph{Code Organization}

Code should be organized according to information hiding and
\texttt{static} modifiers as follows (this is enforced by CheckStyle when using the coding standard's associated rules):
%
\begin{itemize}
\item All fields are declared before all constructors.
\item All constructors are declared before all other methods.
\item All \texttt{static} fields are declared before all
  non-\texttt{static} fields.
\item All \texttt{static} methods are declared before all
  non-\texttt{static} methods.
\item All inner and nested classes are declared after all fields and
  methods.
\item Among the same type of entity (constructors, instance methods,
  static methods, instance fields, static fields), the order for
  information hiding modifiers should be \texttt{public}, then
  \texttt{protected}, then package/default,\footnote{You generally
    don't want to use the default access modifier anyway.} then
  \texttt{private}.
\end{itemize}

\emph{Rationale:} Having all classes organized in the same consistent fashion makes it easy to find any particular type of construct when reading the code, and also makes code review easier by grouping the constructs that need different ``kinds'' of review (e.g., \texttt{public} APIs may have different review criteria than \texttt{private} helper methods).

\paragraph{Miscellaneous}

The following typical coding issues are automatically detected by the
tool configurations that accompany this standard:
%
\begin{itemize}
\item Too many exit points from methods. A method should typically
  have only one exit point (that is, zero or one \texttt{return}
  statements). The exception to this rule is \texttt{equals} methods,
  where it is very common to short-circuit the return value.
\item Nested \texttt{if} statements, or an abundance of
  \texttt{if}/\texttt{case} statements in a single method. In general,
  this means that your logic needs to be reworked into something
  simpler and perhaps broken into multiple methods.
\item Exceptionally long constructors and methods.
\item Exceptionally long classes. 
\item Magic numbers (numeric constants used outside of \texttt{static
    final} declarations). In general, you should declare symbolic
  constants when necessary (e.g., \texttt{public static final int}
  \texttt{NUM\_PRIMARY\_COLORS = 3}). However, you should \emph{never}
  declare symbolic constants of the form ``\texttt{public static final
    int SIX = 6}'' to appease the static checking system; such constants
  do not make the code any easier to understand and will definitely be
  caught by your instructor.
\item Magic strings (string constants that are used more than once in
  the same source file).
\end{itemize}

The static checkers detect many other coding issues as well; in some
cases, their suggestions can be ignored, but in most cases you will
want to follow them. Use your best judgement.


\subsection{Eiffel}

Our base set of Eiffel style guidelines are those documented in \emph{Eiffel: The Language (ETL)} and \emph{Object-Oriented Software Construction}, both by Bertrand Meyer. In particular, Appendix A (Style Guidelines, pps. 483-496) of ETL contains an excellent, to-the-point summary of good Eiffel style.

As discussed in the semantic properties paper, properties are used to augment these style guidelines. They are embedded in Eiffel code through a regularized use of structured Eiffel comments and indexing clauses.

File-scoped comments are documented using indexing clauses as normal for Eiffel code. Modules in Eiffel are classes, so module-scoped specification should be located in the class comment region, just after the class declaration, but before the inherit region of an Eiffel class. Feature and variable structured comments using semantic properties are located in the standard places for feature Eiffel comments, since variables are features---just after the feature declaration.

\section{References}
These guidelines are based on example coding standards from several sources including, but not limited to:
\begin{itemize}
  \item The Caltech Computer Science Infospheres Java Coding Standard, from which this document evolved
  \item Doug Lea's \href{https://gee.cs.oswego.edu/dl/html/javaCodingStd.html}{Java Coding Standard}
  \item \href{https://ambysoft.com/essays/javacodingstandards.html}{Scott Ambler's Java Coding Standards}
  \item Watts Humphrey's book \emph{A Discipline for Software Engineering}, Addison-Wesley, 1995
  \item Mark Fussell's Java Development Standards
  \item The Coding Standards Repository (was hosted at Berkeley) for various languages
  \item Coding Standards for C, C++, and Java by Vision 2000 CCS Package and Application Team
  \item Kent Sandvik's Java Coding Style Guidelines
  \item Taligent's Java Cookbook for porting C++ to Java
  \item Naval Postgraduate School Java Style Guide
  \item The Solo Software Engineering home page
  \item Christiaan Balke, Kors Bos, David Candlin, and Steve Fisher's Eiffel Style Guide
\end{itemize}

\end{document}
